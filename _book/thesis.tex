% This is a LaTeX thesis template for Monash University.
% to be used with Rmarkdown
% This template was produced by Rob Hyndman
% Version: 6 September 2016

\documentclass{monashthesis}

%%%%%%%%%%%%%%%%%%%%%%%%%%%%%%%%%%%%%%%%%%%%%%%%%%%%%%%%%%%%%%%
% Add any LaTeX packages and other preamble here if required
%%%%%%%%%%%%%%%%%%%%%%%%%%%%%%%%%%%%%%%%%%%%%%%%%%%%%%%%%%%%%%%

\author{Nicholas S Spyrison}
\title{Dynamic visualization of high-dimensional data via low-dimension
projections and sectioning across 2D and 3D display devices}
\degrees{B.Sc. Statistics, Iowa State University}
\def\degreetitle{Doctor of Philosophy}
% Add subject and keywords below
\hypersetup{
     %pdfsubject={The Subject},
     %pdfkeywords={Some Keywords},
     pdfauthor={Nicholas S Spyrison},
     pdftitle={Dynamic visualization of high-dimensional data via low-dimension
projections and sectioning across 2D and 3D display devices},
     pdfproducer={Bookdown with LaTeX}
}


\bibliography{thesisrefs}

\begin{document}

\pagenumbering{roman}

\titlepage

{\setstretch{1.2}\sf\tighttoc\doublespacing}

\chapter*{Acknowledgements}\label{acknowledgements}
\addcontentsline{toc}{chapter}{Acknowledgements}

I would like to thank \dots

\chapter*{Declaration}\label{declaration}
\addcontentsline{toc}{chapter}{Declaration}

I hereby declare that this thesis contains no material which has been
accepted for the award of any other degree or diploma in any university
or equivalent institution, and that, to the best of my knowledge and
belief, this thesis contains no material previously published or written
by another person, except where due reference is made in the text of the
thesis.

\vspace*{2cm}\par\authorname

\chapter*{Preface}\label{preface}
\addcontentsline{toc}{chapter}{Preface}

The material in Chapter \ref{ch:intro} has been submitted to
\emph{Something interesting jornal} for possible publication.

The contribution in Chapter \ref{ch:spinifex} of this thesis was
presented in the super awesome confrence held in Dublin, Ireland, in
July 2015.

\chapter*{Abstract}\label{abstract}
\addcontentsline{toc}{chapter}{Abstract}

This thesis is about \ldots{}

\clearpage\pagenumbering{arabic}\setcounter{page}{0}

\chapter{Introduction}\label{ch:intro}

This is where you introduce the main ideas of your thesis, and an
overview of the context and background.

In a PhD, Chapter 2 would normally contain a literature review.
Typically, Chapters 3--5 would contain your own contributions. Think of
each of these as potential papers to be submitted to journals. Finally,
Chapter 6 provides some concluding remarks, discussion, ideas for future
research, and so on. Appendixes can contain additional material that
don't fit into any chapters, but that you want to put on record. For
example, additional tables, output, etc.

\chapter{Literature review}\label{ch:lit_review}

\section{Touring}\label{sec:tour}

\subsection{Overview}\label{overview}

In univariate data sets histograms, or smoothed density curves are
employed to visualize data. In bivariate data scatterplots and contour
plots (2-d density) can be employed. In three dimensions the two most
common techniques are: 2-d scatter plot with the 3rd variable as an
aesthetic (such as, color, size, height, \(etc.\)) or rendering the data
in a 3-d volume using some perceptive cues giving information describing
the seeming depth of the image
\footnote{Graphs of data depicting 3 dimension are typically printed on paper, or rendered on a 2-d monitor, they are intrinsically 2-d images. They are sometimes referred to as 2.5-d, or more frequently erroneously referred to as 3-d, more on this later.}.
When there are 4 variables: 3 variables as spatial-dimensions and a 4th
as aesthetic, or a scatterplot matrix consisting of 4 histograms, and 6
unique combinations of bivariate scatterplots.

Let \(p\) be the number of numeric variables; how do we visualize data
for even modest values of \(p\) (say 6 or 12)? It's far too common that
visualizing in data-space is dropped altogether in favor of modeling
parameter-space, model-space, or worse: long tables of statistics
without visuals \autocite{wickham_visualizing_2015}. Yet, we all know of
the risks inherent in relying too heavily on parameters alone
\autocites{anscombe_graphs_1973}{matejka_same_2017}. So why do we move
away from visualizing in data-space? Scalability, in a word, we are not
familiar with methods that allow us to concisely depict and digest
\(p \geq 5\) or so dimensions. This is where dimensionality reduction
comes in. Specifically, we will be focusing on a specific group called
touring. In the interest of time I will not belabor the diversity of
dimensionality reduction, (see
{[}\textcite{grinstein_high-dimensional_2002};
\textcite{carreira-perpinan_review_1997}; heer\_tour\_2010{]} for a
quick summary). Suffice it to say that touring has a couple of salient
features: linear transformations such that we can interpolate back to
the original variable space and does not discard dimensions, something
that is common to other linear techniques. By employing the breadth of
tours we are able to preserve the visualization of data-space, and with
it, the intrinsic understanding of structure and distribution of data
that is more succinct or beyond the reach of statistic values alone.

Touring is a linear dimensionality reduction technique that orthogonally
projects \(p\)-space down to \(d(\leq p)\) dimensions. Many such
projections are interpolated, each making local rotations in
\(p\)-space. These frames are then viewed in order to the effect of
watching an animation of the lower dimensional embedding changing as
\(p\)-space is manipulated. Shadow puppets offer a useful analogy to aid
in conceptualizing touring. Imagine a fixed light source facing a wall.
When a hand or puppet is introduced the 3-dimensional object projects a
2-dimensional shadow onto the wall. This is a physical representation of
a simple projection, that from \(p=3\) down to \(d=2\). If the object
rotates then the shadow correspondingly changes. Observers watching only
the shadow are functionally watching a 2-dimensional tour as the
3-dimensional object is manipulated.

\subsubsection{Terminology}\label{terminology}

n, p (sometimes called d by Wegman, or n ), d (sometimes called k by
wegman, or d in tourr)

\subsection{History}\label{history}

Touring was first introduced by Asimov in 1985 with his purposed Grand
Tour\autocite{asimov_grand_1985} at Stanford University. In which,
Asimov suggested three types of Grand Tours: torus, at-random, and
random-walk. The specifics of which will be discussed below in the
Typology section.

TALK ABOUT maths Here::

Note that the the above methods have no input from the user aside from
the starting basis. The bulk of touring development since has largely
been around dynamic display, user interaction, geometric representation,
and application.

This works well when the number of dimensions being toured is small (in
the neighborhood of 5-10), yet the number of view, or 2-frames and we
can produce from \(p\)-space suffers from the so called blessing/curse
of dimensionality. In which the plethora of degrees of freedom either
offer many (non-unique) solutions to a problem or something that becomes
ever increasing unlikely,

\subsection{Tour path}\label{tour-path}

A fundamental aspect of touring is the path of rotation. Of which there
are four primary distinctions\autocite{buja_computational_2005}: random
choice, precomputed choice, data driven, and manual control.

\begin{itemize}
\item
  \emph{grand tour}, a constrained random choice \(p\)-space. Paths are
  constrained for changes in direction small enough to maintain
  continuity and aid in user comprehension

  \begin{itemize}
  \tightlist
  \item
    torus-surface \autocite{asimov_grand_1985}
  \item
    Geodesic
  \item
    at-random
  \item
    random-walk
  \item
    \emph{local tour}, a sort of grand tour on leash, such that it goes
    to a nearby random projection before returning to the original
    position and iterating
  \end{itemize}
\item
  \emph{guided tour}, data driven tour optimizing some objective
  function via (stochastic) gradient descent
  \autocite{hurley_analyzing_1990}.

  \begin{itemize}
  \tightlist
  \item
    holes \autocite{cook_projection_1993} - iterates projections that
    add more white space to the center of the projection.
  \item
    cmass \autocite{cook_projection_1993} - find the projection with the
    most density or mass in the center.
  \item
    lda \autocite{lee_projection_2005} - linear discriminant analysis,
    seeks a projection where 2 or more classes are most separated.
  \item
    pda - principal component analysis finding where the data is most
    spread (1d only).
  \item
    other user-defined objective function \autocite{wickham_tourr_2011}.
  \end{itemize}
\item
  \emph{planned tour}, Precomputed choice, In which the path has already
  been generated or defined.

  \begin{itemize}
  \tightlist
  \item
    \emph{little tour} \autocite{mcdonald_interactive_1982}, where every
    permutation of variables is stepped through in order, analogous to a
    brute-force or exhaustive search.
  \item
    a saved path of any other tour
  \end{itemize}
\item
  \emph{manual tour} - Manual control, a constrained rotation on
  selected manipulation variable and
  magnitude\autocite{cook_manual_1997}. Typically used to explore the
  local area after identifying an interesting feature from another tour.
\item
  \emph{dependance tour}, combination of \(n\) independent 1d tours. A
  vector describes the axis each variable will be displayed on.
  \textbf{ie} \(c(1, 1, 2, 2)\) is a 4 to 2d tour with the first 2
  variables on on the first axis, and the remaining on the second.

  \begin{itemize}
  \tightlist
  \item
    \emph{correlation tour} \autocite{buja_data_1987}, a special case of
    the dependence tour, analogous to canonical correlation analysis
  \end{itemize}
\end{itemize}

\subsection{Geometrics and display
dimension}\label{geometrics-and-display-dimension}

Up to this point we have been talking about 2d scatterplots, which offer
the first and a simple case for viewing lower-dimensional embeddings of
\(p\)-space. However, other geometrics (or geoms) offer perfectly valid
orthonormal projections as well.

\begin{itemize}
\tightlist
\item
  1d geoms

  \begin{itemize}
  \tightlist
  \item
    1-d densities: such as histogram, average shifted
    histograms\autocite{scott85}, and kernel density\autocite{scott95}.
  \item
    image: \autocite[ 2001]{Wegman}
  \item
    time series: where multivariate values are independently lagged to
    view peak and trough alignment. Currently no package implementation,
    but use case is discussed in \autocite{cook_manual_1997}.
  \end{itemize}
\item
  2d geoms

  \begin{itemize}
  \item
    2-d density \autocite[ GITHUB]{NS}
  \item
    scatterplot
  \item
  \end{itemize}
\item
  2.5d, 3d geoms \{ADD FOOTNOTE ABOUT 2.5d vs 3d\}

  \begin{itemize}
  \tightlist
  \item
    Anaglyphs, sometimes called stereo, where (typically) red images are
    positioned for the left channel and cyan for the right, when viewed
    with corresponding filter glasses give the depth perception of the
    image.
  \item
    Depth, which use some subset of depth cues, most commonly size
    and/or color of data points.
  \end{itemize}
\item
  \(d\)-dim geoms

  \begin{itemize}
  \tightlist
  \item
    Andrews curves \autocite{andrews_plots_1972}, smoothed variant of
    parallel coordinate plots, discussed below.
  \item
    Chernoff faces \autocite{chernoff_use_1973}, variables linked to
    size of facial features for rapid cursory like-ness comparison of
    observations.
  \item
    Parallel coordinate plots \autocite{ocagne_coordonnees_1885}, where
    any number of variables are plotted in parallel with observations
    linked to their corresponding variable value by polylines.
  \item
    Scatterplot matrix \autocite{becker_brushing_1987}, showing a
    triangle matrix of bivariate scatterplots with 1-d density on the
    diagonal.
  \item
    Radial glyphs, radial variants of parallel coordinates including
    radar, spider, and star glyphs \autocite{siegel_surgical_1972}.
  \end{itemize}
\end{itemize}

\subsection{Aplication}\label{aplication}

Below is a non-exhaustive list of software implementing touring in some
degree, ordered by descending year:

\begin{itemize}
\tightlist
\item
  Spinifex \autocite{spinifex} -- for Linux, Unix, and Windows.
\item
  Tourr \autocite{wickham_tourr_2011} -- for Linux, Unix, and Windows. R
  package.
\item
  CyrstalVision \autocite{wegman_visual_2003} -- for Windows.
\item
  GGobi \autocite{swayne_ggobi:_2003} -- for Linux and Windows.
\item
  DAVIS \autocite{huh_davis:_2002} -- Java based, with GUI.
\item
  VRGobi \autocite{nelson_xgobi_1998} -- for use with the C2 in
  stereoscopic 3d displays.
\item
  ExplorN \autocite{carr_explorn:_1996} -- for SGI Unix.
\item
  XGobi \autocite{swayne_xgobi:_1991} -- for Linux, Unix, and Windows
  (via emulation).
\item
  XLispStat \autocite{tierney_lisp-stat:_1990} -- for Unix, and Windows.
\item
  Prim-9 \autocites{asimov_grand_1985}{fisherkeller_prim-9:_1974} -- on
  an internal operating system.
\end{itemize}

Support and maintenance of such implementations give them a particularly
short life span, while conceptual abstraction and technically heavier
implementations have hampered user growth. There have been notable
efforts to diminish the barriers to entry and make touring more
approachable as a data exploration tool {[}\textcite{huh_davis:_2002};
\textcite{swayne_ggobi:_2003}; \textcite{wegman_visual_2003};
\textcite{wickham_tourr_2011}; huang\_tourrgui:\_2012{]}.

\section{Virtual reality}\label{virtual-reality}

\chapter{\texorpdfstring{\emph{spinifex}: An R package that provides
manual rotations in
high-dimensions}{spinifex: An R package that provides manual rotations in high-dimensions}}\label{ch:spinifex}

\section{Abstract}\label{abstract-1}

The tour algorithm, and its various versions provide a systematic
approach to viewing low-dimensional projections of high-dimensional
data. It is particularly useful for understanding multivariate data, and
useful in association with techniques for dimension reduction,
supervised and unsupervised classification. The \emph{R} package
\emph{tourr} provides many methods for conducting tours on multivariate
data. This paper discusses an extension package which adds support for
the manual tour, called \emph{spinifex}. It is particularly usefully for
exploring the sensitivity of structure discovered in a projection by a
guided tour, to the contribution of a variable. \emph{Spinifex} utilizes
the animation packages \emph{plotly} and \emph{gganimation} to allow
users to rotate a variable into and our of a chosen projection.

Keywords: grand tour, projection pursuit, manual tour, high dimensional
data, multivariate data, data visualization, statistical graphics, data
science, data mining.

\section{Introduction}\label{introduction}

A tour is a multivariate data analysis technique in which is a sequence
of orthogonal projection into a lower subspace are viewed in order. each
frame of the sequence corresponds to a small change in the projection
for a smooth transition.

Multivariate data analysis can be broken into 2 groups: linear and
non-linear transformations. Similar to PCA and LDA, touring uses linear
dimension reduction with inter-operability back to the original
parameter-space. They differ from non-linear transformations such as
t-SNE (t-distributed stochastic nearest neighbor embeddings), MDS
(multi-dimension scaling), and LLE (local linear embedding), which
distort parameter-space for more opaque interpretations

There are many ways that a tour path can be generated, we will focus on
one in particular, the manual tour. The manual tour was described in
\textcite{cook_manual_1997}, and allows a user to rotate a variable into
and out of a 2D projection of high-dimensional space. The primary
purpose is to determine the sensitivity of structure visible in a
projection to the contributions of a variable. Manual touring can also
be useful for exploring the local structure once a feature of interest
has been identified, for example, by a guided tour
\autocite{cook_grand_1995}. The algorithm for a manual tour allows
rotations in horizontal, vertical, oblique, angular and radial
directions. Rotation in a radial direction, would pull a variable into
and out of the projection, which allows for examining the sensitivity of
structure in the projection to the contribution of this variable. This
type of manual rotation is the focus of this paper.

A manual tour relies on user input, and thus has been difficult to
program in R. Ideally, the mouse movements of the user are captured, and
passed to the computations, driving the rotation interactively. However,
this type of interactivity is not simple in R. This has been the reason
that the algorithm was not incorporated into the \emph{tourr} package.
Spinifex utilizes two new packages for conducting animations,
\emph{plotly} \autocite{sievert_plotly_2018} and \emph{gganimate}
\autocite{pedersen_gganimate:_2019}, to conduct a manual tour. From a
given projection, the user can choose which variable to control, and the
animation sequence is generated to remove the variable from the
projection, and then extend its contribution to be the sole variable in
one direction. This allows the viewer to assess the change in structure
induced in the projection by the variable contribution.

The paper is organized as follows. Section \ref{sec:algorithm} explains
the algorithm using a toy dataset. Section \ref{sec:application}
illustrates how this can be used for sensitivity analysis. The last
section summarizes the work and discusses future research.

\section{Algorithm}\label{sec:algorithm}

Creating a manual tour animation requires these steps:

\begin{enumerate}
\def\labelenumi{\arabic{enumi}.}
\tightlist
\item
  Provided with a 2D projection, choose a variable to explore. This is
  called the ``manip'' variable.
\item
  Create a 3D manipulation space, where the manip variable has full
  contribution.
\item
  Generate a rotation sequence which zero's the norm of the coefficient
  and also increases it to 1.
\end{enumerate}

These steps are described in more detail below.

\subsection{Notation}\label{notation}

This section describes the notation used in the algorithm description.
The data to be displayed is an \(n \times p\) numeric matrix.

\begin{align*}
  \textbf{X}_{n \times p} ~=
  \begin{bmatrix}
    X_{1,~1} & \dots  & X_{1,~p} \\
    X_{2,~1} & \dots  & X_{2,~p} \\
    \vdots   & \ddots & \vdots   \\
    X_{n,~1} & \dots  & X_{n,~p}
  \end{bmatrix}
\end{align*}

and an orthonormal \(d\)-dimensional projection matrix is

\begin{align*}
  \textbf{B}_{[p,~d]} ~=
  \begin{bmatrix}
    B_{1,~1} & \dots  & B_{1,~d} \\
    B_{2,~1} & \dots  & B_{2,~d} \\
    \vdots   & \ddots & \vdots   \\
    B_{p,~1} & \dots  & B_{p,~d}
  \end{bmatrix}
\end{align*}

The algorithm is primarily operating on the projection basis and
utilizes the data only when making a display.

\subsection{Toy data set}\label{toy-data-set}

The flea data from the R package \emph{tourr}
\autocite{wickham_tourr_2011}, is used to illustrat the algorithm. The
data, originally from \textcite{lubischew_use_1962},contains 74
observations across 6 variables, which physical measurements of the
insects. Each individual belonged to one of three species.

A guided tour on the flea data is conducted by optimizing on the
\texttt{holes} index \autocite{cook_interactive_2007}. In a guided tour
the data the projection sequence is shown by optimizing an index of
interest. The holes index is maximized by when the projected data has a
lack of observations in the center. Figure \ref{fig:step0}, shows an
optimal projection of this data. The left plot displays the projection
basis, while the right plot shows the projected data. The display of the
basis has a unit circle with lines showing the horizontal and vertical
contributions of each variable in the projection. Here is is primarily
tars1 and aede2 contrasting the other four variables. In the projected
data it can be seen that there are three clusters, which have been
colored, although not used in the optimization. The question that will
be explored in the explanation of the algorithm is how important is
aede2 to the separation of the clusters.

\begin{figure}
\centering
\includegraphics{thesis_files/figure-latex/step0-1.pdf}
\caption{\label{fig:step0}Basis reference frame (left) and projected data
(right) of standardized flea data. Basis identified by holes-index
guided tour. The variables \texttt{aede2} and \texttt{tars1} contribute
mostly in the x direction, whereas the other variables contribute mostly
in the y direction. We'll select \texttt{aede2} as our manipulation
variable to see how the structure of the projection changes as we rotate
\texttt{aede2} into and out of the projection.}
\end{figure}

The left frame of \ref{fig:step0} shows the reference frame for the
basis. It describes the X and Y contributions of the basis as it
projects from the 6 variable dimensions down to 2. Call
\texttt{view\_basis()} on a basis to produce a similar image as a
\texttt{ggplot2} object. The right side shows how the data looks
projected through this basis. You can project a single basis at any time
through the matrix multiplication
\(\textbf{X}_{[n,~p]} ~*~ \textbf{B}_{[p,~d]} ~=~ \textbf{P}_{d[n,~d]}\)
to such effect.

\subsection{Step 1 Choose variable of
interest}\label{step-1-choose-variable-of-interest}

Select a manipulation variable, \(k\). Initialize a zero vector \(e\),
and set the \(k\)-th element set to 1.

\begin{align*}
\textbf{e}_{k~[p,~1]} ~=~ 
  \begin{bmatrix}
    0 \\
    0 \\
    \vdots \\
    1 \\
    \vdots \\
    0
  \end{bmatrix}_{[p,~1]}
\end{align*}

In figure \ref{fig:step0}, above, notice that the variables
\texttt{tars1} and \texttt{aede2} are almost orthogonal to the other 4
variables and control almost all of the variation in the x axis of the
projection. \texttt{Aede2} has a larger contribution in this basis, so
we'll select it

\subsection{Step 2 Create the manip
space}\label{step-2-create-the-manip-space}

Use the Gram-Schmidt process to orthonormalize the concatenation of the
basis and \(e\) yielding the manipulation space.

\begin{align*}
  \textbf{M}_{[p,~d+1]}
  &= Orthonormalize_{GS}( \textbf{B}_{[p,~d]}|\textbf{e}_{k~[p,~1]} ) \\
  &= Orthonormalize_{GS}
  \left(
    \begin{bmatrix}
      B_{1,~1} & \dots  & B_{1,~d} \\
      B_{2,~1} & \dots  & B_{2,~d} \\
      \vdots   & \ddots & \vdots   \\
      B_{k,~1} & \dots  & B_{k,~d} \\
      \vdots   & \ddots & \vdots   \\
      B_{p,~1} & \dots  & B_{p,~d}
    \end{bmatrix}
  ~|~
    \begin{bmatrix}
      0 \\
      0 \\
      \vdots \\
      1 \\
      \vdots \\
      0
    \end{bmatrix}
  \right)
\end{align*}

In R it looks like the below chunk. \texttt{tourr::orthonormalise()}
uses the Gram Schimidt process (rather than Householder reflection) to
orthonormalize.

\begin{Shaded}
\begin{Highlighting}[]
\NormalTok{  e            <-}\StringTok{ }\KeywordTok{rep}\NormalTok{(}\DecValTok{0}\NormalTok{, }\DataTypeTok{len =} \KeywordTok{nrow}\NormalTok{(basis))}
\NormalTok{  e[manip_var] <-}\StringTok{ }\DecValTok{1}
\NormalTok{  manip_space  <-}\StringTok{ }\NormalTok{tourr}\OperatorTok{::}\KeywordTok{orthonormalise}\NormalTok{(}\KeywordTok{cbind}\NormalTok{(basis, e))}
\end{Highlighting}
\end{Shaded}

Adding an extra dimension to our basis plane allows for the manipulation
of the specified variable while the others are kept fully within the
basis plane. orthonormalizing rescales the matrix without bringing the
other variables into this new axis. An illustration of such can been
seen below in \ref{fig:step2}.

\begin{figure}
\centering
\includegraphics{thesis_files/figure-latex/step2-1.pdf}
\caption{\label{fig:step2}Manipulation space for controlling the
contribution of aede2 of standardized flea data. Basis was identified by
holes-index guided tour. The out of plane axis, in red, shows how the
manipulation variable can be rotated, while other dimensions stay
embedded within the basis plane.}
\end{figure}

Imagine being able to grab hold of the red axis and rotate it changing
the projection onto the basis plane. This is what happens in a manual
tour. By controlling the angle between the axis and the basis plane we
change the contribution of the manipulation variable on the projection.

\subsection{Step 3 Generate rotation}\label{step-3-generate-rotation}

Define a set of values for \(\phi_i\), the angle of out-of plane
rotation, orthogonal to the projection plane. This corresponds to the
angle between the red manipulation axis and the blue plane in
\ref{fig:step2}.

\textbf{For } \(i\) \textbf{in 1 to n\_slides:}

For each \(\phi_i\), post multiply the manipulation space by a rotation
matrix, producing as many basis-projections.

\begin{align*}
  \textbf{P}_{b[p,~d+1,~i]}
  &= \textbf{M}_{[p,~d+1]} ~*~ \textbf{R}_{[d+1,~d+1]} 
    ~~~~~~~~~~~~~~~~~~~~~~~~\text{For the $d=2$ case:} \\
  &= \begin{bmatrix}
    M_{1,~1} & \dots & M_{1,~d} & M_{1,~d+1} \\
    M_{2,~1} & \dots & M_{2,~d} & M_{2,~d+1} \\
    \vdots   & \ddots& \vdots   \\
    M_{p,~1} & \dots & M_{p,~d} & M_{p,~d+1}
  \end{bmatrix}_{[p,~d+1]}
    ~*~
  \begin{bmatrix}
    c_\theta^2 c_\phi s_\theta^2 &
    -c_\theta s_\theta (1 - c_\phi) &
    -c_\theta s_\phi \\
    -c_\theta s_\theta (1 - c_\phi) &
    s_\theta^2 c_\phi + c_\theta^2 &
    -s_\theta s_\phi \\
    c_\theta s_\phi &
    s_\theta s_\phi &
    c_\phi
  \end{bmatrix}_{[3,~3]}
\end{align*}

Where:

\begin{description}
  \item[$\theta$] is the angle that lies on the projection plane ($ie.$ on the XY plane)
  \item[$\phi$] is the angle orthogonal to the projection plane ($ie.$ in the Z direction)
  \item[$c_\theta$] is the cosine of $\theta$
  \item[$c_\phi$]   is the cosine of $\phi$
  \item[$s_\theta$] is the sine of   $\theta$
  \item[$s_\phi$]   is the sine of   $\phi$
\end{description}

In application: compile the sequence of \(\phi_i\) and create an array
(or long table) for each rotated manipulation space. \(\phi\) is
actually the angle relative to the \(\phi_1\), we find the
transformation \(\phi_i\) - \(\phi_1\) useful to discuss \(\phi\)
relative to the basis plane.

\begin{Shaded}
\begin{Highlighting}[]
\ControlFlowTok{for}\NormalTok{ (phi }\ControlFlowTok{in} \KeywordTok{seq}\NormalTok{(seq_start, seq_end, phi_inc_sign)) \{}
\NormalTok{  slide <-}\StringTok{ }\NormalTok{slide }\OperatorTok{+}\StringTok{ }\DecValTok{1}
\NormalTok{  tour[,, slide] <-}\StringTok{ }\KeywordTok{rotate_manip_space}\NormalTok{(manip_space, theta, phi)[, }\DecValTok{1}\OperatorTok{:}\DecValTok{2}\NormalTok{]}
\NormalTok{\}}
\end{Highlighting}
\end{Shaded}

In \ref{fig:step3} we illustrate the sequence with 15 projected bases
and highlight the manip variable on top, while showing the corresponding
projected data points on the bottom. A dynamic version of this tour can
be viewed online at
\url{https://nspyrison.netlify.com/thesis/flea_manualtour_mvar4/}, will
take a moment to load. This format of this figure and linking to dynamic
version will be used again the \ref{sec:application} section.

\begin{figure}
\centering
\includegraphics{thesis_files/figure-latex/step3-1.pdf}
\caption{\label{fig:step3}Rotated manipulation spaces, a radial manual tour
manipulating aded2 of standardized flea data. The manipulation variable,
aede2, extends from it's initial contribution to a full contribution to
the projection before decreasing to zero, and then returning to it's
initial state. A dynamic version can be viewed at
\url{https://nspyrison.netlify.com/thesis/flea_manualtour_mvar4/}.}
\end{figure}

\section{Display projection sequence}\label{display-projection-sequence}

To get back to data-space pre-multiply each projection basis by the data
for the projection in data-space.

\begin{align}
  \textbf{P}_{d[n,~d+1]}
    &= \textbf{X}_{[n,~p]} ~*~ \textbf{P}_{b[p,~d+1]} \\
    &= 
      \begin{bmatrix}
          X_{1,~1} & \dots & X_{1,~p} \\
          X_{2,~1} & \dots & X_{2,~p} \\
          \vdots   & \vdots & \vdots  \\
          X_{n,~1} & \dots & X_{n,~p}
      \end{bmatrix}_{[n,~p]}
      ~*~
      \begin{bmatrix}
        P_{b:1,~1} & P_{b:1,~2} & P_{b:1,~3} \\
        P_{b:2,~1} & P_{b:2,~2} & P_{b:2,~3} \\
        \vdots     & \vdots     & \vdots     \\
        P_{b:p,~1} & P_{b:p,~2} & P_{b:p,~3}
      \end{bmatrix}_{b[p,~d+1]}
\end{align}

Plot the first 2 variables from each projection in sequence for an XY
scatterplot. The remaining variable is sometimes linked to a data point
aesthetic to produce depth cues used in conjunction with the XY
scatterplot.

\emph{tourr} utilizes R's base graphics for the display of tours. Use
\texttt{render\_plotly()} to display as an dynamic \texttt{plotly}
\textcite{sievert_plotly_2018} object or \texttt{render\_gganimate()}
for a \texttt{gganimate} \textcite{pedersen_gganimate:_2019} graphic. A
third notable animation related package is \texttt{animation}
\textcite{xie_animation:_2018}. It's not yet implemented in
\texttt{spinifex} as it uses base graphics, whereas the former two are
compatible with \texttt{ggplot2}.

Interaction with graphics in R is limited. Traditionally, all commands
are passed to the R via calls to the console, conflicting with user
engagement. Some recent packages have made advancement into this
direction such as with the use of the R package \texttt{shinny}, which
custom-made applications can be hosted either locally or remotely and
interact with the R console, allowing for developers to code dynamic
content interaction. To a lesser extent \texttt{plotly} offers static
interactions with contained object, such as tool tips, brushing, and
linking without communicating back to the R console.

Storing the each data point and all of the overhead though goes into
dynamic graphics if very inefficient. In the same way that we performed
math the bases, that is the same approach storage and sharing tours.
Consider the manual tour, we can store the salient features in 3 basis,
where \(\phi\) is at it's starting, minimum, and maximum values. The
frames in between can be interpolated by supplying angular speed or
number of desired frames. By using the \texttt{tourr::save\_history()}
we can do just that. Save such tour path history and a single set of the
data offers a performant storage and transferring.

\section{Application}\label{sec:application}

In a recent paper, \textcite{wang_visualizing_2018}, the authors
aggregate and visualize the sensitivity of hadronic experiments. The
authors introduce a new tool, PDFSense, to aid in the visualization of
parton distribution functions (PDF). The parameter-space of these
experiments lies in 56 dimensions, \(\delta \in \mathbb{R}^{56}\), and
are presented in this work in 3-d subspaces of the 10 first principal
components and non-linear embeddings.

The work in \textcite{cook_dynamical_2018} applies touring for discern
finer structure of this sensitivity. Table 1 of Cook et. al. summaries
the key findings of PDFSense \& TFEP (tensorflow embedded projection)
and those from touring. The authors selected the 6 first principal
components, containing 48\% of the variation held within the full data
when centered, but not sphered. This data contained 3 clusters: jet,
DIS, and VBP. Below pick up from the projections used in their figures 7
and 8 (jet and DIS clusters respectively) and apply manual tours to
explore the local structure with finer precision.

\subsection{Jet cluster}\label{jet-cluster}

The jet cluster is of particular interest as it contains the largest
data sets and is found to be important in
\textcite{wang_visualizing_2018}. The jet cluster resides in a smaller
dimensionality than the full set of experiments with 4 principal
components explaining 95\% of it's variation
(\textcite{cook_dynamical_2018}). We subset the data down to ATLAS7old
and ATLAS7new to narrow in on 2 groups with a reasonable number of
observations and occupy different parts of the subspace. Below, we
perform radial manual tours on various principal components within the
this scope. In PC3 and PC4 are manipulated in \ref{fig:JetClusterGood}
and \ref{fig:JetClusterBad} respectively. Manipulating PC3, where
varying the angle of rotation brings interesting features in-to and
out-of the center mass of the data, is interesting than the manipulation
of PC4, where features are mostly independent of the manip var.

\begin{figure}
\centering
\includegraphics{thesis_files/figure-latex/JetClusterGood-1.pdf}
\caption{\label{fig:JetClusterGood}Jet cluster, radial manual tour of PC3.
Colored by experiment type: `ATLAS7new' in black and `ATLAS7old' in red.
When PC3 fully contributes to the projection ATLAS7new (black) occupies
unique space and several outliers are identifiable. Zeroing the
contribution from PC3 to the projection hides the outliers and indeed
all observations with ATLAS7new are contained within ATLAS7old (red). A
dynamic version can be viewed at
\url{https://nspyrison.netlify.com/thesis/jetcluster_manualtour_pc3/}.}
\end{figure}

\begin{figure}
\centering
\includegraphics{thesis_files/figure-latex/JetClusterBad-1.pdf}
\caption{\label{fig:JetClusterBad}Jet cluster, radial manual tour of PC4.
Colored by experiment type: `ATLAS7new' in black and `ATLAS7old' in red.
This tour contain less interesting information ATLAS7new (black) has
points that are right and left of ATLAS7old, while most points occupy
the same projection space, regardless of the contribution of PC4. A
dynamic version can be viewed at
\url{https://nspyrison.netlify.com/thesis/jetcluster_manualtour_pc3/}.}
\end{figure}

Jet cluster manual tours manipulating each of the principal components
can be viewed from the links below:

\begin{itemize}
\tightlist
\item
  PC1 -
  \url{https://nspyrison.netlify.com/thesis/jetcluster_manualtour_pc1/}
\item
  PC2 -
  \url{https://nspyrison.netlify.com/thesis/jetcluster_manualtour_pc2/}
\item
  PC3 -
  \url{https://nspyrison.netlify.com/thesis/jetcluster_manualtour_pc3/}
\item
  PC4 -
  \url{https://nspyrison.netlify.com/thesis/jetcluster_manualtour_pc4/}
\end{itemize}

\subsection{DIS cluster}\label{dis-cluster}

We perform a manual tour on this data, manipulating PC6 as depicted in
\ref{fig:DISclusterGood}. Looking at several frames we see that DIS HERA
lie mostly on a plane. When PC6 has full contributions we see the dimuon
SIDIS in green is almost orthogonal to the DIS HERA (black). Yet the
contribution of PC6 is zeroed the dimuon SIDIS data occupy the same
space as the DIS HERA data. A dynamic version of this manual tour can be
found at:
\url{https://nspyrison.netlify.com/thesis/discluster_manualtour_pc6/}.
The page take a bit to load, as the animation is several megabytes.

\begin{figure}
\centering
\includegraphics{thesis_files/figure-latex/DISclusterGood-1.pdf}
\caption{\label{fig:DISclusterGood}DIS cluster, radial manual tour of PC6.
colored by experiment type: `DIS HERA1+2' in black, `dimuon SIDIS' in
green, and `charm SIDIS' in red. When the contribution PC 6 is large we
see that dimuon SIDIS (green) data are nearly orthogonal to DIS HERA
(black) data. As the data is rotated, we can also see that DIS HERA
(black) practically lie on a plane in this 6-d subspace. When the
contribution of PC6 is near zero, dimonSIDIS (green) occupies the same
space as the DIS HERA data. A dynamic version can be viewed at
\url{https://nspyrison.netlify.com/thesis/discluster_manualtour_pc6/}.}
\end{figure}

This is different story than if we had selected a different variable to
manipulate. In \ref{fig:DISclusterBad} we manipulate PC2.

\begin{figure}
\centering
\includegraphics{thesis_files/figure-latex/DISclusterBad-1.pdf}
\caption{\label{fig:DISclusterBad}DIS cluster, radial manual tour of PC2.
Colored by experiment type: `DIS HERA1+2' in black, `dimuon SIDIS' in
green, and `charm SIDIS' in red. The structure of previously described
plane of DIS HERA (black) and nearly orthogonal dimuon SIDIS (green) is
present, however the manipulating PC2 does not give a head-on view of
either, a less useful manual tour than that of PC6. A dynamic version
can be viewed at
\url{https://nspyrison.netlify.com/thesis/discluster_manualtour_pc2/}.}
\end{figure}

DIS cluster manual tours manipulating each of the principal components
can be viewed from the links below:

\begin{itemize}
\tightlist
\item
  PC1 -
  \url{https://nspyrison.netlify.com/thesis/discluster_manualtour_pc1/}
\item
  PC2 -
  \url{https://nspyrison.netlify.com/thesis/discluster_manualtour_pc2/}
\item
  PC3 -
  \url{https://nspyrison.netlify.com/thesis/discluster_manualtour_pc3/}
\item
  PC4 -
  \url{https://nspyrison.netlify.com/thesis/discluster_manualtour_pc4/}
\item
  PC5 -
  \url{https://nspyrison.netlify.com/thesis/discluster_manualtour_pc5/}
\item
  PC6 -
  \url{https://nspyrison.netlify.com/thesis/discluster_manualtour_pc6/}
\end{itemize}

\section{Source code and usage}\label{source-code-and-usage}

This article was created \texttt{bookdown}
(\textcite{xie_bookdown:_2016}) using \texttt{rmarkdown}
(\textcite{xie_r_2018}), with code generating the examples inline, and
the source files can be found at
\href{https://github.com/nspyrison/confirmation/}{github.com/nspyrison/confirmation/}.

The source code for the \texttt{spinifex} package can be found at
\href{https://github.com/nspyrison/spinifex/}{github.com/nspyrison/spinifex/}.
To install the package in R, run:

\begin{Shaded}
\begin{Highlighting}[]
\CommentTok{# install.package("devtools")}
\NormalTok{devtools}\OperatorTok{::}\KeywordTok{install_github}\NormalTok{(}\StringTok{"nspyrison/spinifex"}\NormalTok{)}
\end{Highlighting}
\end{Shaded}

\section{Discussion}\label{discussion}

The work in the package can be extended in a number of ways: new
rotation mechanisms, generalize manual touring to other geoms with
scalable dimension, dynamic capture and control of the manipulation axis
for oblique manual tours.

The Givens rotations and Householder reflections as outlined in
\textcite{buja_computational_2005} could be implemented. in Givens
rotations, the x and y components (\(ie. \theta~= 0,pi/2\)) of the
in-plane rotation are calculated separately and applied sequentially
rather than radially. Householder reflections define reflection axes to
project project points on to the axes.

\emph{tourr} offers a number of \(d\)-dimensional graphic displays
including andrews curves, chernoff faces, parallel coordinate plots,
scatterplot matrix, and radial glyphs. Care should be take to extend
graphics such that they are agnostic to rotation type and tour path.

a \emph{shiny} app or other such front-end could aid in the user
interaction, offering initialization to starting projections by use of
guided tours, as we did in \ref{fig:step0} and manip variable selection.
A method to transfer from current basis to the manipulation space of the
new manip var would be a nice addition.

\chapter{Display dimensionality}\label{ch:disp_dim}

\begin{itemize}
\tightlist
\item
  XGobbi vs the C2
\end{itemize}

\chapter{Human-computer interaction of 3d
projections}\label{ch:hci_3dproj}

\begin{itemize}
\tightlist
\item
  Tour in 3D
\item
  ImAxes / IATK
\end{itemize}

\printbibliography[heading=bibintoc]



\end{document}
